\documentclass[conference]{IEEEtran}
\IEEEoverridecommandlockouts
% The preceding line is only needed to identify funding in the first footnote. If that is unneeded, please comment it out.
\usepackage{cite}
\usepackage{amsmath,amssymb,amsfonts}
\usepackage{algorithmic}
\usepackage{graphicx}
\usepackage{textcomp}
\usepackage{xcolor}
\usepackage{subcaption}
\def\BibTeX{{\rm B\kern-.05em{\sc i\kern-.025em b}\kern-.08em
    T\kern-.1667em\lower.7ex\hbox{E}\kern-.125emX}}
\begin{document}

\title{Flight Test of a Collision Avoidance Neural Network with Run-Time Assurance
%{\footnotesize \textsuperscript{*}Note: Sub-titles are not captured in Xplore and
%should not be used}
%\thanks{Identify applicable funding agency here. If none, delete this.}
}

\author{\IEEEauthorblockN{Darren Cofer, Ramachandra Sattigeri, \\
Isaac Amundson, Junaid Babar, \\
Saqib Hasan}
\IEEEauthorblockA{\textit{Collins Aerospace} \\
first.last@collins.com}
\and
\IEEEauthorblockN{Eric Smith, \\
Karthik Nukala}
\IEEEauthorblockA{\textit{Kestrel Institute} \\
\{eric.smith, nukala\}@kestrel.edu}
\and
\IEEEauthorblockN{Denis Osipychev, Lucca Timmerman, \\
Dragos Margineantu, James Paunicka, \\
Matthew Moser}
\IEEEauthorblockA{\textit{Boeing} \\
\{first.last\}@boeing.com}
}

\maketitle

\begin{abstract}
Aircraft systems have requirements for airworthiness certification that present barriers to the use
of machine learning technologies such as neural networks. Showing that a component or system is
correct and does no harm through behaviors that were unintended by designers or unexpected by
operators is a critical aspect of the certification process. Our team is developing assurance
technologies that can support the use of machine learning in the design of safety-critical aircraft
systems. These capabilities have been integrated on Boeing’s autonomy demonstrator aircraft to show
that they can provide evidence of correct operation and safety guarantees needed by real aircraft.
We have applied run-time assurance along with formal methods modeling and analysis tools to an
airborne collision avoidance system based on a neural network. This system was demonstrated
in-flight and shown to correctly monitor neural network operation and intervene when needed to
prevent violation of the “stay well clear” safety requirement relative to an intruder aircraft.
\end{abstract}

\begin{IEEEkeywords}
machine learning, run-time assurance
\end{IEEEkeywords}

\section{Introduction}

Aircraft systems have requirements for airworthiness certification that present barriers to the use
of machine learning technologies such as neural networks. Showing that a component or system is
correct and does no harm through behaviors that were unintended by designers or unexpected by
operators is a critical aspect of the certification process. In a typical machine learning
application, much of the complexity and design information resides in its training data rather than
in the actual models or software produced. This means that it is generally not possible to determine
the correctness of a {\em learning enabled component (LEC)}, such as a neural network, 
by examining its implementation or tracing specific design
elements back to requirements.

Our team is developing assurance technologies that can support the use of machine learning in the
design of safety-critical aircraft systems. These capabilities have been integrated on Boeing's
autonomy demonstrator aircraft to show that they can provide evidence of correct operation and
safety guarantees needed by real aircraft. In previous work \cite{dasc2020} we have used a run-time assurance
(RTA) architecture to ensure the safety of an autonomous neural network-based aircraft taxiing
application. In our current work we have applied RTA along with formal methods
modeling, synthesis, and analysis tools to an airborne collision avoidance system based on a neural network.
This system was demonstrated in-flight and shown to correctly monitor neural network operation and
intervene when needed to prevent violation of the ``remain well clear'' safety requirement relative to
an intruder aircraft. Thirty-six test conditions were flown with various collision course
geometries
%, with relative heading angles of 45, 90, 135, 180 (head-on), 225, 270, and 315 degrees
%leading to a pending collision, 
to test the robustness of the neural net trajectory generator and
the RTA mitigations.

The RTA architecture includes a run-time monitor that provides an independent assessment of the
avoidance flight plans generated by the neural network and a safe (but less optimal) backup planner.
The results of the assessment are evaluated by a decision logic component which selects (based on a
tabular specification of safety rules) a flight plan that will ensure safe flight. The core decision
logic code is synthesized from a formal specification, with most of the synthesis steps producing
machine-checked proofs of their correctness. The RTA architecture has been modeled in the
Architecture Analysis and Design Language (AADL) and formally analyzed to show that it maintains the
system safety requirements.

Remain-well-clear and collision avoidance capabilities are critical to safe flight of autonomous
military and commercial aircraft. These capabilities can also be valuable for
enhancing the safety of flight for piloted aircraft by providing pilot cueing. Run-time assurance
approaches, with run-time monitoring of machine learning software functions integrated with
contingency management functionality, will enable safe use of neural networks and enable new
autonomous capabilities for aircraft.

In this paper we will discuss: 
\begin{itemize} 
\item The assurance challenges associated with the use of LECs in safety-critical aircraft applications 
\item The autonomy framework and aircraft used to demonstrate the collision avoidance neural network capability 
\item The run-time assurance architecture developed to guarantee that any potential unintended behaviors in the neural network do not impact safety
\item The formal methods assurance technologies applied within the architecture, including analysis of the architecture design and synthesis of decision logic from a formal specification
\item Flight testing, results obtained for various test scenarios, and lessons learned from the
demonstration
\end{itemize}

% motivation
% aerospace applications
% barriers - requirements, verification, coverage and unintended functionality
% brief description of the demo application and clear statement of our accomplishments

\section{Assurance Challenges}

[DARREN - 0.5 pg]

The assurance challenges for the use of neural networks in safety-critical aircraft applications

Show absence of unintended functionality -- do no harm

For software in commercial aircraft, DO-178C \cite{DO-178} provides
the latest version of guidance regarding software aspects of
certification and is used by the aviation industry and regulators
as a means of compliance with airworthiness regulations.
At a high level, DO-178C helps manufacturers achieve two
main goals: 1) to demonstrate that software complies with
its requirements (intended functionality), and 2) that it does
nothing unexpected (unintended functionality). Unintended
functionality or unintended behavior can therefore be defined
as software behavior than cannot be traced back to any requirement.

LECs and their software implementations
break many of the assumptions that are the basis for current
certification processes. In particular, the design intent cannot
be inferred from an examination of an LEC model or its
software implementation.

LECs present unique challenges that may be barriers to the
use of traditional, model-based, or formal methods guidance
currently defined in DO-178C and its supplements. Fundamentally,
this is due to the reliance on requirements-based testing
(or verification) and structural coverage metrics.  

Structural coverage metrics were constructed with the understanding
that much of the complexity of traditional software
is manifested in the logical decisions that are being implemented.
This logic should be traceable to specific software
requirements. When requirements-based tests fail to exercise
part of the software logic as revealed by structural coverage
metrics, it is reasonable to conclude that something is amiss
(either a missing requirement or some unintended function).
Since neural networks do not primarily implement logical
decisions, structural coverage can usually be achieved with one
test case (or possibly a small number of them) \cite{whitebox}. Individual lines
of code in the software representation of a neural network do
not represent design choices that implement specific requirements.
Therefore, current structural coverage metrics are not
helpful in identifying unintended behaviors.

Since it is difficult to demonstrate assurance by examining
the LEC design (as is assumed by existing certification processes),
other approaches based on run-time monitoring and
enforcement may be effective. 
% Assurance Challenges
% The assurance challenges for the use of neural networks in safety-critical aircraft applications [DARREN]
% show absence of unintended functionality -- "do no harm"

\section{Autonomy Framework}

Our experiments are carried out on Boeing's Autonomy Testbed aircraft and simulation environment.  This testbed is used for the development of the collision avoidance LEC, the RTA architecture for ensuring safety, and for flight testing and demonstration.  

\subsection{Autonomy Testbed Aircraft}

%[MATT/JIM] Overview of the Boeing autonomy framework and aircraft used to demonstrate the collision avoidance neural network capability.

%Collision avoidance problem -- Strategic rather than tactical, avoidance flight plan must avoid intruder but return to original flight plan

%Limitations --  two-dimensional, lateral avoidance maneuvers, focus on single intruder

%Describe the safety requirements for collision avoidance.  Maybe reference DO-365. 

Our experiments are carried out using the Boeing Autonomy Testbed Aircraft, a Cessna Caravan 208B, tail number N208BX.  This platform is currently serving as a test bed for the DARPA Assured Autonomy program air domain experiments (\ref{fig:caravan}).  The Testbed is optionally piloted and serves as a means to demonstrate commercially viable technologies leading to autonomy.  It is a research and development vehicle able to operate in commercial airspace that is built on open-source middleware with in-house developed guidance and control technologies leveraged from across the Boeing enterprise.  The Testbed includes a full ``Iron Bird'' fixture for hardware-in-the-loop evaluation in which new autonomy technology can be fully integrated and tested before flight.  With this Testbed Boeing has demonstrated autonomy firsts including in-air detect and avoid, ADS-B transponder-based route planning for strategic avoidance, and fully autonomous ground taxi.

\begin{figure}
	\centering
	\includegraphics[width=\columnwidth]{figures/caravan.jpg}
	\caption{Boeing Autonomy Testbed Aircraft - Cessna Caravan 208B}
	\label{fig:caravan}
\end{figure}

Detect and avoid (DAA) mission operating and performance standards are defined in RTCA document DO-365 \cite{DO_365}.  The standard provides guidance for interactions of Unmanned Aircraft Systems (UAS) in the National Airspace System, requirements for safe operation of aircraft during encounters including separation distance minimums for remaining well clear of aircraft and avoicing mid-air collisions, and proper aircraft equipage to achieve safe detect and avoid operations.
The assurance challenged posed in our work focuses on the Autonomy Testbed aircraft flying in the vicinity of another ``intruder'' airplane, where the test flight software includes a Boeing-developed LEC to generate an avoidance flight plan for the Testbed to remain well clear of the intruder aircraft as defined in DO-365.  The underlying assurance technology montors the Boeing LEC in order to assess the avoidance trajectory from the LEC and guarantee safety. 

Figure \ref{fig:rta-arch} shows a block diagram of the autonomy system framework as deployed on the Testbed aircraft.
System elements include: 
\begin{itemize}
\item ADS-B – Primary sensor for perceiving the airspace is the Automated Dependent Surveillance Broadcast (ADS-B) system providing detection information on  nearby aircraft (intruders) to the other system functions.
%\item ADS-B Tracker – Generates intruder track information from received ADS-B messages to the other system functions.
\item Avoidance Alerts – Evaluates potential future traffic conflicts and issues “alerts” to the Avoidance functions.  Assessment definition and requirements are specified in DO-365 MOPS for DAA System.
\item LEC Planner -- Neural network system trained offline through reinforcement learning to generate avoidance flight plans satisfying DAA requirements.  Training and opertaion is described in the following subsection.  
\item Backup Avoidance Planner – A trusted but less optimal backup planners that provides waypoint navigation paths to avoid airspace incursion.  Avoidance computation method is virtual predictive radar which is designed to provide maximum ``safe passage timed corridors.''  Avoidance path terminates back on the original flight plan.
\item Run-Time Assurance  – Run-time monitoring, predication, and assessment funcation that guarantee the selection of a safe flight plan for the aircraft.  Operation and assurance are described in \ref{sec:rta} and \ref{sec:assurance}.
\item Autonomous Executive – Constructs and manages execution of the vehicle flight plan and contains a function to ``splice in'' avoidance guidance waypoint plans into the original flight plan.
%\item Vehicle Manager – Executes flight path provided by AE including guidance and control for the vehicle.  The VM also sends commands and receives feedback from actuation system components.
%\item Actuators / Sensors – Carries out VMS flight control surface commands and provides positional feedback.
%\item Ownship State Estimation – Provides vehicle state information including position, altitude, and speed along with the vehicle inertial reference frame.
%\item Navigation Database – Reference database of aircraft parameters and airspace waypoints, terrain, airports, approaches, etc.  Used for route/avoidance planning purposes.
\end{itemize}

\begin{figure*}
	\centering
	\includegraphics[width=\textwidth]{figures/rta-arch.jpg}
	\caption{Run-Time Assurance Architecture for Collision Avoidance on the Autonomy Testbed Aircraft}
	\label{fig:rta-arch}
\end{figure*}
% Matt's section

\subsection{LEC Training with Reinforcement Learning}
%Description of LEC and RL, ROS integration

The LEC that generates the collsion avoidance flight plan is trained offline with a 
%In this work, the corrective maneuver for the avoidance is generated by a pre-computed solution -- 
Reinforcement Learning (RL) policy model. 
RL is a sequential policy optimization method that solves the task using the ``learning by doing'' concept and requires continuous data-rich interaction with the environment \cite{sutton2018reinforcement}. 
Our avoidance policy is trained on a surrogate task to provide the large number of interactions needed to achieve good performance. 
This surrogate environment developed by Boeing is a lightweight Python environment integrated with the OpenAI GYM framework \cite{brockman2016openai}.

The learned policy minimizes the risk of collision by providing continuous control commands in the surrogate environment. These commands are converted into a geometric trajectory (a sequence of waypoints) using the surrogate environment and a Robot Operating System (ROS) interface (Figure~\ref{fig:diagram}).

\begin{figure}[h]
	\centering
	\includegraphics[width=\linewidth]{figures/system_overview.pdf}
	\caption{System diagram of the RL model-based conflict resolver.}
	\label{fig:diagram}
\end{figure}


Our surrogate environment is a simplified 2D obstacle avoidance problem that mimics the real task (air traffic conflict resolution). 
The environment simulates the movements of two aircraft on a 20x20 km square. The controlled Agent (representing the Autonomy Testbed Aircraft) has to go around the Intruder, provide minimal horizontal separation, and merge back to the next safe waypoint from the original route before the simulation ends.

%\begin{table}[h]
%	\centering
%	\begin{tabular}{||c | c | c||} 
%		\hline
%		Parameter & Range & Units \\
%		\hline
%		Time step $dT$ & 1.0  & sec \\
%		Max time $t_\text{MAX}$ & 300.0  & sec \\
%		Distance range & -10000 .. 10000  & m \\
%		Horizontal separation & 1000  & m \\
%		\hline
%	\end{tabular}
%	\caption{Surrogate environment simulation parameters.}
%	\label{tbl:sim_param}
%\end{table}


The surrogate dynamics imitate the dynamics of the Autonomy Testbed Aircraft.
%vehicle selected for the experiment -- a single-engine turboprop Cessna 208 Grand Caravan.
Both aircraft are represented by a simple dynamic model assuming massless kinematics and using Dubin's vehcile model for turn dynamics.
%\begin{itemize}
%	\item Dubin's vehicle model for turn dynamics
%	\item Mass-less kinematics model 
%	\end {itemize}
All the observation and control values are normalized to [-1..1] for the RL agent.
To make sure the agent generalizes the problem, we rewrap the observations and focus only on 
relative positions rather than absolute.  Repacked observations consumed by the agent:
	\begin{center}
		\begin{tabular}{|| c | c ||} 
			\hline
			heading & intruder heading\\
			airspeed & intruder airspeed\\
			distance to goal & distance to intruder\\
			tracking angle to goal & tracking angle to intruder\\
			\hline
		\end{tabular}
	\end{center}

\subsection{RL Policy Agent}

The RL policy agent learns the task by interacting with the simulation and iteratively updating the parameters of the policy model using Stochastic Gradient Descent (SGD) optimization. We approximate the policy with a multi-layered perceptron shown in Fig. \ref{fig:policy_model}.

\begin{figure}[h]
	\centering
	\includegraphics[width=0.8\columnwidth]{figures/model.pdf}
	\caption{Neural network function approximation used for the policy model consists of 2 hidden layers, 256 neurons each.}
	\label{fig:policy_model}
\end{figure}

To solve the optimization problem as Markov Decision Process (MDP), we refactor it into Markovian states $s$, transitions  $T(s' | s, a)$, and transition reward $R(s' | s, a)$. 
The state of the system (including both Agent and Intruder) is fully observable, assumes the perfect knowledge, and is enough to describe the Markovian state of the MDP system.
The state of the agent is described as
$$ s = \{ v_a, \psi_a, v_i, \psi_i, \beta_i, d_i, \beta_g, d_g \}$$

where 
$v_a$ is agent speed,
$\psi_a$ is agent heading,
$v_i$ is intruder speed,
$\psi_i$ is intruder heading,
$\beta_i$ is angle to intruder,
$d_i$ is distance to intruder,
$\beta_g$ is angle to goal, and
$d_g$ is distance to goal. 

The optimization is set to find the optimal policy $\pi^*(s)$ as a set of state-action mappings that maximizes the expected reward $V(s)$ \cite{sutton2018reinforcement}.
\begin{align} 
	\pi(s) &= P(a | s) \\
	\pi^*(s) &= \arg\max_{\pi} V^\pi (s) \\
	&= \arg\max_a \left( R(s,a) + \gamma T(s'|s,a) V(s') \right)
\end{align} 

The value of the state is the expected future reward accumulated over the trajectory and defined by the Bellman function as:
\begin{align}
	V(s) &=  \mathbb{E} [R | s, \pi] \\
	&= \sum_{s'} T(s'|s,a) \left( R(s'|s,a) + \gamma ( V(s')) \right) \\
	&=  R(s'|s,a) + \gamma \sum_{s'} T(s'|s,a) V^{\pi}(s') \\
	V^{*}(s) &= \max_{a} \left( R(s,a) + \gamma \sum_{s'} T(s'|s,a) V^{*}(s') \right)
\end{align}

The RL policy model is based on the Actor-Critic architecture that helps to improve the stability of the training \cite{sutton2018reinforcement}. The SGD-based update for Actor $\theta$ and Critic $w$ network is:
\begin{align}
	\delta &=  R_{t+1} +\gamma \hat V(s_{t+1},w) - \hat V(s_t,w) \\
	w &\leftarrow w + \alpha \delta \nabla \hat V (s, w) \\
	\theta &\leftarrow \theta + \alpha \delta \nabla \ln \pi (a|s, \theta)
\end{align} 

The core functionality of the RL agent incorporates the Stable Baselines library, a very reputable fork of OpenAI Baselines \cite{hill2018stable}. For the exploration policy and update steps, this work used the Proximal Policy Optimization (PPO) algorithm that becomes the state of the art in continuous-action agents \cite{schulman2017proximal}.


\subsection{LEC Integration on Testbed Aircraft}

\begin{figure}[h]
	\centering
	\includegraphics[width=\linewidth]{figures/cp25_ros.pdf}
	\caption{System diagram of the ROS-LEC node showing the integration of the learning-enabled component (LEC) to the aircraft using the ROS interface.}
	\label{fig:integration}
\end{figure}

System integration is done using the Robot Operating System (ROS) interface \cite{quigley2009ros}. This allows unifying the interfaces to the high-fidelity simulation and to the physical aircraft demonstrator.
The ROS-LEC Agent, shown in Figure~\ref{fig:integration}, aggregates data from different domains and provides important utilities to the system. Its job is designed as follows: 
\begin{itemize}
	\item receive and accumulate ROS messages regarding the own-ship state,
	Intruder's state, traffic alerts, GPS-SRS transformation data,
	\item translate ADS-B and GPS positioning data to local coordinate frame,
	\item extract the goal location from the original route,
	\item re-wrap the observations into the Agent-specific input format,
	\item iteratively run the Agent to get the corrective actions,
	\item iteratively run the surrogate environment to receive the transitions,
	\item form a corrective trajectory and check if the trajectory is good,
	\item translate the trajectory from local coordinate frame to global lat-long waypoints,
	\item publish the trajectory as a ROS message.
\end{itemize}

On an external request, the ROS-LEC agent generates a single avoidance flight plan and publishes it as ROS message. The flight plan consists of 20 waypoints in total. The last waypoint is taken from the original route, and 19 waypoints are generated by the policy. This allows linking the waypoints by a unique index and presere the indices of the original route.

These waypoints are spaced 20 seconds apart which provides a 400-second planning horizon. 
Because of the large time step between the waypoints, the policy and the surrogate have to be evaluated 20 times to make a single waypoint. The total response time of the system is below 60 msec for the complete trajectory.

%When needed to re-plan, the runner can be requested again. This architecture allows closed-loop corrections with an external run-time assurance monitor. The monitor keeps track of the accumulated transition error and either requests an updated avoidance plan or denies the operation switching to a back-up mode.

% Autonomy Framework [MATT/DENIS]
% The autonomy framework and aircraft used to demonstrate the collision avoidance neural network capability.
% Safety requirements
% Reinforcement learning and the LEC produced for collision avoidance.
% Execute LEC repetitively to generate avoidance trajectory.

\section{Run-Time Assurance}
\label{sec:rta}

[DARREN - 1.5 pg]

Run-Time Assurance Architecture -- ASTM F3269-17 standard [DARREN]

Run-time assurance architectures add high-assurance components
to the system to ensure that an LEC cannot cause
unsafe or unintended system behaviors. Run-time monitors
continuously check variables related to the system state, inputs
to the LEC, or outputs produced by the LEC and intervene to
switch to a backup function that is proven to be safe. Monitors
may be used to detect anomalous inputs that are outside of the
data distribution used to train the system and therefore could
lead to unintended behavior. The main idea is that system
performance is provided by the LEC while system safety is
guaranteed by high-assurance components (though with lower
performance).

In the DARPA Assured Autonomy project we have used a
run-time assurance architecture based on the ASTM F3269-17
standard for bounded behavior of complex systems \cite{F3269-17}, also
known as a simplex architecture \cite{simplex}. The standard provides
guidance for mitigating unintended functionality through the
use of run-time monitors. The LEC may still contain unintended
functionality, but the architecture ensures that there will
be no impact on system safety. This approach essentially uses
the verified properties of the architecture, run-time monitor,
and safety backup functions to justify a reduced level of
criticality for the LEC.

Monitors, high-assurance components, backup avoidance plan

Run-time assurance architecture was developed to guarantee the absence of unintended behavior resulting from the collision avoidance neural network

%- RTA, AADL model [DARREN/ISAAC]

The run-time assurance architecture is illustrated in Figure~\ref{fig:rta-arch}.  Incoming ADS-B messages are assessed by the Detect \& Avoid (DAA) subsystem.  When an avoidance alert is generated, the Safe Backup Planner generates a baseline avoidance function (BAF) plan, which then triggers the generation of an LEC plan.  The RTA monitor performs a stay well clear (SWC) assessment on both plans in the context of the current ownship state, and the Plan Selector then chooses one of the two plans based and informs the Plan Switch of its decision.
The Plan Switch publishes a flight plan based on the Plan Selector output.

\subsection{Trajectory prediction} 
Trajectory prediction over a defined prediction horizon (in time units) consists of prediction of both the intruder and own-ship trajectories based on underlying assumptions of the intruder's future velocity and the own-ship's ability to track the avoidance flight plans from the LEC and the Safe Backup Planner. For the application in consideration, the prediction horizon was set to 180 seconds under the assumption that the avoidance flight plans produced would be frozen for this time period. In cases where the avoidance plans can be generated more frequently, the prediction horizon can appropriately be reduced which should improve the accuracy of intruder trajectory prediction and allow for more confidence in the SWC evaluation.

For the intruder, each incoming ADS-B message is processed as a measurement to a tracking filter designed as per Appendix D in \cite{DO_366A}. The tracking filter assumes that the intruder aircraft is either in a constant velocity (CV) mode or in a constant speed turn (CT) mode. The multiple-model tracking filter is designed to be robust to basic errors in the ADS-B message data (such as data with too small or large uncertainties), missing data or repeat data, but mostly the ADS-B data is considered a trustworthy source of information of the intruder 3D position. The tracking filter produces estimates of the intruder current position and velocity along with their respective uncertainties, and with a prediction of the specific mode of the intruder behavior (CV/CT). The trajectory prediction propagates the estimate from the tracking filter forward in time by assuming the aircraft continues in its current mode (CV/CT) with a growth model in the longitudinal velocity uncertainty and a cross-track position uncertainty, with the cross-track position uncertainty capped to a maximum value. The uncertainties in the intruder trajectories can be visualized as ellipses whose semi-major/semi-minor axes are aligned with the longitudinal/normal to velocity vector and whose lengths are proportional to the uncertainties along/normal to the velocity vector. These assumptions allow to account for decaying confidence over longer prediction horizons.

The own-ship trajectory predictions along the avoidance flight plans assume a kinematic model of the own-ship behavior using performance parameters such as maximum bank angle, roll-rate, longitudinal acceleration, etc. The trajectory prediction assumes that the first waypoint of the avoidance flight plans lies directly in the path of the current velocity vector. The own-ship trajectory consists of constant ground speed segments between waypoints and constant speed arcs connecting line segments between consecutive waypoints. As the actual aircraft is flying constant airspeed, and not ground speed, deviations between the predicted and actual aircraft trajectories are expected. These deviations are accounted for by a fixed tracking error bound that is configurable. The tracking error bound $+$ vehicle size parameter are padded onto the specified SWC separation distance to produce a conflict radius threshold around the own-aircraft.

The trajectory prediction models are run internally at a high-rate ($>= $ 10 Hz) for both the intruder and own-ship, but the output trajectories are sampled at 1 Hz to produce time-stamped discretized trajectory points which are evaluated for conflicts as discussed in the next sub-section.

\subsection{SWC assessment} 
The SWC assessment consists of evaluating time-stamped samples of the predicted trajectory of the intruder and the own-aircraft trajectory along each produced avoidance flight plan. The evaluation considers the intersection of the uncertainty ellipse around an intruder predicted trajectory sample with the conflict circle around the corresponding own-aircraft predicted trajectory sample. The probability of intersection is determined using the analytical approximations documented in Section IV of \cite{prob_conflict_detection}. If the probability exceeds a configurable threshold, then the flight plan is determined to be unsafe. 

In addition to the boolean determination of safety, the point to closest approach (CPA) between the own-ship and intruder predicted trajectories is calculated along with the time to CPA. The CPA metrics allow the Decision Logic Component to choose between plans that are both marked unsafe by the SWC assessment. These metrics allow for a comparison with the actual measured CPA metrics during simulation and test flights as a way to evaluate the run-time monitor performance.
% Run-Time Assurance Architecture, monitors, high-assurance components, ASTM F3269-17 standard
% The run-time assurance architecture developed to guarantee the absence of unintended behavior resulting from the neural network
% - RTA, AADL model [DARREN/ISAAC]
% - Trajectory prediction, SWC assessment [RAM/DARREN]

\section{Assurance Technologies}

[DARREN - 3 pg total for section]

The formal methods assurance technologies applied within the architecture, 
including analysis of the architecture design and synthesis of decision logic from a formal specification

% Subsections
% - AGREE
% - Resolute
% - Logic

% Assurance Technologies
% The formal methods assurance technologies applied within the architecture,
% including analysis of the architecture design and synthesis of decision logic from a formal specification

\subsection{Architecture Verification}

%AGREE analysis of RTA arch in AADL

One of the steps for design-time assurance of the RTA architecture is verifying the architecture satisfies its high-level requirements.  Traditionally, requirements verification has been achieved using a combination of directed testing methods and manual review. However, model-based specification enables a more rigorous approach to verification via formal methods analysis. With both the requirements and architecture represented in formal (well-defined, unambiguous) notations, satisfiability modulo theories (SMT) solvers can be employed to determine whether there is any possible sequence of inputs that will violate a requirement.  Furthermore, failure by the solver to find a counterexample is essentially equivalent to a mathematical proof that the requirement can never be violated.  

In architectural models, we can represent high-level requirements as assume-guarantee contracts on components.  \textit{Guarantees} are statements about a component's outputs which will always hold as long as stated \textit{assumptions} are valid.  When designing an architecture that includes multiple components, it is imperative to verify that a system's subcomponent contracts satisfy the overall system contract, as well as whether a component's assumptions are valid with respect to the specified upstream guarantees and the environment.
%
We use the Assume Guarantee Reasoning Environment (AGREE)~\cite{agree2012} to specify and analyze component contracts in our run-time assurance architecture.  AGREE is a plugin for the Open Source AADL Tool Environment (OSATE), enabling contracts to be specified directly on AADL model components and analyzed within the modeling environment.

%The RTA architecture was modeled in AADL (Figure~\ref{fig:rta-agree}) using the Open Source AADL Tool Environment (OSATE), and the high-level requirements were added to this AADL model as AGREE specifications \cite{agree2012}.  AGREE specifications are used to describe component behavior through assume-guaranteee contracts, and these contracts are analyzed using AGREE's compositional reasoning framework available through the AGREE plugin for OSATE.

The main objective of our AGREE analysis of the collision avoidance system was to verify that the RTA architecture is guaranteed to publish only safe flight plans.
%
%In Figure~\ref{fig:rta-agree}, when an avoidance alert is generated, the Safe Backup Planner generates a BAF plan; this triggers the generation of an LEC plan. The Plan Selector chooses one of the two plans based on the SWC Assessment output, and informs the Plan Switch of its decision. The Plan Switch publishes a flight plan based on the Plan Selector output.
As illustrated in Figure~\ref{fig:rta-agree}, we annotated the AADL model with assume-guarantee contracts for each component in the architecture, and AGREE was able to verify the high-level property under the assumption that the backup avoidance flight plan is always safe.
There are four guarantees shown in Figure~\ref{fig:rta-agree}:
\begin{itemize}
\item An ADS-B intruder conflict results in a BAF plan being generated.
\item An ADS-B intruder conflict results in an LEC plan being generated.
\item An ADS-B intruder conflict results in a safe plan being generated by the High Assurance System under the assumption that the BAF plan is always safe.
\item An ADS-B intruder conflict results in a safe plan being selected by the Plan Switch under the assumption that the BAF plan is always safe.
\end{itemize}

The figure also demonstrates the use of AGREE to detect {\em vacuous} guarantees.  This refers to guarantees that are true only because the context in which the guarantee should hold never occurs. Figure~\ref{fig:rta-agree} shows two such statements:
\begin{itemize}
\item The High Assurance System always selects the BAF plan if it is safe.
\item The Plan Switch always selects the BAF plan if it is safe.
\end{itemize}
In other words, we are checking to see if the RTA architecture ever selects the LEC plan.  AGREE analysis shows that these statements are false, giving legitimacy to the guarantees regarding the High Assurance System and Plan Switch.

\begin{figure*}
	\centering
	\includegraphics[width=\textwidth]{figures/rta-agree-v2.jpg}
	\caption{AADL Model of Run-Time Assurance Architecture and AGREE properties}
	\label{fig:rta-agree}
\end{figure*}

% - AGREE analysis of RTA arch in AADL [JUNAID/ISAAC]

\subsection{Decision Logic Synthesis}



We formalized this decision table and applied our synthesis tools, based on the
ACL2 theorem prover~\cite{acl2}, to synthesize high-assurance decision code
implementing the table logic.  This process used the $\verb+def-table+$
machinery described in~\cite{dasc2020}.  In particular, proofs are
automatically performed on the table to ensure completeness (some case always
applies, and an output is always produced) and unambiguity (no more than one
case applies, so a unique output is always produced).

Our software synthesis process starts with an executable specification that applies our generic table evaluation procedure \texttt{apply-table} to the specific table described above.  In this initial specification, the table is represented as a piece of data that is interpreted by \texttt{apply-table}.  Synthesis steps will specialize the table evaluation process, building in the specific table to create a cascade of conditional expressions.

We synthesize an executable Java program in two key steps:
\begin{enumerate}

\item Apply APT transformations to specialize and simplify the
  decision logic, creating an optimized, executable function: We use Kestrel's
  APT (Automated Program Transformations) \cite{apt} toolkit, built on ACL2,
  especially the \texttt{simplify} transformation, which applies simplifying
  rewrite rules from our library.
  %% Other synthesis steps
  %% include applications of our \texttt{letify}, and
  %% \texttt{lift-condition} transformations.
  %% \texttt{Letify} introduces \texttt{LET} constructs to bind variables and
  %% avoid re-computation, and \texttt{lift-condition} brings if-then-else
  %% expressions to the top level.
  Each of the APT transformation steps produces a proof showing the equivalence
  of its input and output.

\item Generate Java code using ATJ\cite{atj}: We use Kestrel's ATJ (ACL2-to-Java) Java
  code generator, built on ACL2, to generate Java code for integration with the
  Collins RTA.  Extending previous work (described in \cite{dasc2020}), ATJ now
  generates idiomatic and more performant Java code.

\end{enumerate}

For integration into the larger system, we created a hand-written wrapper for the generated Java decision code.  The wrapper receives and processes incoming plan messages from the Remain Well Clear Assessment, applies the generated decision code to the boolean plan variables, and publishes the decision to the Plan Switch.  Communication is done via ROS.

Our methodology for synthesizing the decision logic using program transformations allows for quick re-synthesis when the design changes.  One simply re-runs the synthesis tools using the new table-based specification as input.  This flexibility was demonstrated when simulation results necessitated additional conditions and constraints on plan selection.  Specifically, the table was changed to consider whether each plan's tCPA exceeds the 179 second planning horizon. This simply amounted to changing the \texttt{def-table} encoding in ACL2, with all the correctness proofs, synthesis/optimization steps, and Java code generation following immediately.

Future improvements to this methodology may include:
\begin{itemize}
\item Outfitting ATJ with proofs: As remarked in \cite{dasc2020}, this might involve either formalizing the semantics of Java or using Kestrel's Axe toolkit to lift Java bytecode into logic.
\item Code generation for other languages (C, Python, Rust): Recent work at Kestrel was able to achieve the same synthesis presented above using the ATC C code generator\cite{atc} in place of ATJ, with the additional benefit of having proofs-of-correctness for the code generation.
\item Formal synthesis of RTA-logic interface: At the time of publication, the integration of the plan selection logic with the Collins RTA requires a handwritten wrapper, which could potentially also be synthesized with proofs.
\end{itemize}

% - Decision logic, tabular spec and evolution, APT synth and proof [ERIC/KARTHIK]

\subsection{Assurance Argument}

[SAQIB/ISAAC - 1 pg]

Resolute Assurance argument for RTA arch


\begin{figure*}
	\centering
	\includegraphics[width=\textwidth]{figures/rta-resolute.jpg}
	\caption{Assurance Argument for run-time assurance generated by Resolute}
	\label{fig:rta-resolute}
\end{figure*}

% - Resolute Assurance argument for RTA arch [SAQIB/ISAAC]

\section{Flight Test Results}

%[DARREN/JIM - 1.5 pg]
Our flight testing plan evaluated the performance of the LEC collision avoidance planner and the RTA architecture using the Boeing Autonomy Testbed Aircraft and another (unmodified) Cessna Caravan aircraft flying as an intruder.  Both aircraft executed a variety of straight-and-level trajectories headed towards a defined collision point (but with 400 feet of vertical separation for safety) in our test area over Central Washington State.  The flights occurred in an airspace volume closely coordinated with Air Traffic Control at Grant County International Airport in Moses Lake, Washington.  The flight test plan, and the design of all aircraft systems, underwent thorough safety reviews by the US Air Force and the Boeing Company before flight testing.

The two aircraft used in the flight testing were equipped with ADS-B In and ADS-B Out functionality.  The Boeing autonomy testbed aircraft used its onboard ADS-B In system to sense the location of the intruder aircraft which was on a collision course towards the Control Point (CP), as shown in Figure~\ref{fig:flight-test}.  The DAA alerting functionality on the testbed aircraft detected the intruder at a safe distance and triggered the generation of avoidance flight plans that the testbed would fly to remain well clear of the intruder aircraft.  

\begin{figure*}
	\centering
	\includegraphics[width=\textwidth]{figures/flight-test.jpg}
	\caption{Flight test plan for evaluation of LEC collision avoidance and RTA safety guarantees}
	\label{fig:flight-test}
\end{figure*}

As described above, two avoidance flight plans were automatically generated on the testbed aircraft, one by the LEC and the other by the backup avoidance planner (which did not use a neural network).
The RTA architecture assessed the two plans and determined which would be flown by the testbed to remain well clear of the intruder.

Flight testing consisted of the live execution of multiple two-aircraft test conditions.  Each test condition specified the following:
\begin{enumerate}
\item A relative heading angle between the testbed aircraft and the intruder
\item The specification to use one of two LECs that were available on the testbed aircraft flight software
\end{enumerate}

Multiple relative heading angles were flown, including a head-on encounter (relative heading = 180 degrees), and other relative headings in 45 degree intervals.
The available LECs were termed the ``good LEC'' and the ``bad LEC.''  The ``good LEC'' was expected to generate safe remain-well-clear trajectories, while the ``bad LEC'' was designed to generate unsafe trajectories, simulating an LEC producing unintended (and unsafe) behaviors.  

During the numerous test conditions flown we made the following general observations:
\begin{itemize}
\item In test conditions where the good LEC was used, the generated avoidance flight plans resulted in safe and standards conformant remain-well-clear avoidance of the intruder.  The RTA functionality successfully assessed the LEC plans as safe, which resulted in the testbed aircraft flying the LEC route.  
\item In test conditions where the bad LEC was used, the generated avoidance flight plans resulted in violation of the remain-well-clear avoidance criteria relative to the intruder.  The RTA functionality successfully assessed the LEC plans as unsafe, which resulted in the testbed aircraft flying the route from the backup avoidance planner.  
\end{itemize}

%There was a case, though, where the bad LEC actually created a route good enough to be standards conformant, and the Collins RTA functionality did judge that route as safe, which resulted in the testbed aircraft flying the LEC route.

There were two interesting test conditions in which we observed unexpected results.  Recall from Section~\ref{sec:rta} that when both plans are assessed as safe but predicted CPA for either occurs at the limit of the prediction horizon, we prefer the plan whose CPA is actually within the prediction horizon.  This situation occurred spontaneously in one of our test conditions, resulting in the RTA functionality choosing (correctly) the backup plan over the LEC.  

In another test condition we were surprised to observe the RTA functionality choosing to fly the plan generated by the bad LEC.  Upon further analysis, we found that both plans were assessed as safe (though the bad LEC plan was just barely safe) and in this case the plan selector logic correctly chose the LEC plan.  However, during execution of the test scenario the RWC separation criteria was violated.  We discovered that several factors combined to place the intruder aircaft ahead of its predicted position, resulting in separation slightly below the RWC requirement.  This condition was possible in our experiment because of an initial design decision to have each of the planners produce only a single avoidance flight plan at the start of a collision encounter, with no updates for any changes that might occur during test execution.  


%For the conclusion:
%- changes/improvements for future demos (more dynamic)
% Flight Test Results
% Flight testing, results obtained for various test scenarios, and lessons learned from the demonstration [DARREN/JIM]

\section{Conclusion}

%[DARREN - 1 pg, incl refs]
%
%More dynamic RTA architecture, execute/update more frequently
%
%Multiple intruders


Our team has flight tested machine learning technology for aircraft collision avoidance with a
run-time assurance architecture designed to guarantee safety in the presence of potential
unintended behaviors.  These capabilities were integrated on Boeing's
Autonomy Testbed Aircraft to show that they can provide correct operation and
safety guarantees needed by real aircraft.  Flight testing demonstrated the ability of the RTA
system to ensure that ``remain well clear'' separation criteria were maintained during a variety of 
collision encounter geometries.  Formal methods and an assurance argument were used to 
provide evidence of the correctness of the RTA design. 

Future work will extend the LEC planner and the RTA architecture to handle multiple intruder 
aircraft and other obstacles such as weather.  We will also update the RTA architecture to 
actively update RWC assessment during a collision encounter and dynamically request
new avoidance plans if safety has been compromised due to change in intruder aircraft behavior 
or the arrival of additional intruders.  

% next steps
% other applications?

\subsubsection*{Acknowledgment}
This work was funded by DARPA contract  FA8750-18-C-0099. The views, opinions and/or
findings expressed are those of the author and should not be interpreted as representing
the official views or policies of the Department of Defense or the U.S. Government.

\bibliographystyle{abbrv}
\bibliography{biblio}

\end{document}
