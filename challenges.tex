\section{Assurance Challenges}

For software in commercial aircraft, DO-178C \cite{DO-178} provides
the latest version of guidance regarding software aspects of
certification and is used by the aviation industry and regulators
as a means of compliance with airworthiness regulations.
At a high level, DO-178C helps manufacturers achieve two
main goals: 1) demonstrate that software complies with
its requirements (intended functionality), and 2) show that it does
nothing unexpected (unintended functionality). Unintended
functionality or unintended behavior can therefore be defined
as software behavior than cannot be traced back to any requirement.

LECs and their software implementations
break many of the assumptions that are the basis for current
certification processes. An LEC design
is created through training based on large amounts of example
data.  This means, for example, that individual elements
in a neural network (weights, connections, or activation 
functions configured during training) do not represent design 
choices that can be traced back to specific requirements.
Therefore, the design intent cannot
be inferred from an examination of an LEC model or its
software implementation.  

DO-178C fundamentally relies on requirements-based testing
(or verification) and structural coverage metrics, and works 
extremely well to show that a traditional software development process 
correctly implements a set of requirements.  
Structural coverage metrics were constructed with the understanding
that much of the complexity of traditional software
is manifested in the logical decisions that are being implemented.
This logic should be traceable to specific software
requirements. When requirements-based tests fail to exercise
part of the software logic as revealed by structural coverage
metrics, it is reasonable to conclude that something is amiss
(either a missing requirement or some unintended behavior).
Since neural networks do not primarily implement logical
decisions, structural coverage can usually be achieved with one
test case (or possibly a small number of them) \cite{whitebox}. 
Therefore, current structural coverage metrics are not
helpful in identifying and eliminating unintended behaviors. 
Several alternative coverage metrics have been proposed for 
neural networks, but so far none has been shown to provide
equivalent assurance.  

Since it is difficult to demonstrate assurance by examining
the LEC design (as is assumed by existing certification processes),
other approaches based on run-time monitoring and
enforcement may be effective. 