\section{Assurance Challenges}

[DARREN - 0.5 pg]

The assurance challenges for the use of neural networks in safety-critical aircraft applications

Show absence of unintended functionality -- do no harm

For software in commercial aircraft, DO-178C \cite{DO-178} provides
the latest version of guidance regarding software aspects of
certification and is used by the aviation industry and regulators
as a means of compliance with airworthiness regulations.
At a high level, DO-178C helps manufacturers achieve two
main goals: 1) to demonstrate that software complies with
its requirements (intended functionality), and 2) that it does
nothing unexpected (unintended functionality). Unintended
functionality or unintended behavior can therefore be defined
as software behavior than cannot be traced back to any requirement.

LECs and their software implementations
break many of the assumptions that are the basis for current
certification processes. In particular, the design intent cannot
be inferred from an examination of an LEC model or its
software implementation.

LECs present unique challenges that may be barriers to the
use of traditional, model-based, or formal methods guidance
currently defined in DO-178C and its supplements. Fundamentally,
this is due to the reliance on requirements-based testing
(or verification) and structural coverage metrics.  

Structural coverage metrics were constructed with the understanding
that much of the complexity of traditional software
is manifested in the logical decisions that are being implemented.
This logic should be traceable to specific software
requirements. When requirements-based tests fail to exercise
part of the software logic as revealed by structural coverage
metrics, it is reasonable to conclude that something is amiss
(either a missing requirement or some unintended function).
Since neural networks do not primarily implement logical
decisions, structural coverage can usually be achieved with one
test case (or possibly a small number of them) \cite{whitebox}. Individual lines
of code in the software representation of a neural network do
not represent design choices that implement specific requirements.
Therefore, current structural coverage metrics are not
helpful in identifying unintended behaviors.

Since it is difficult to demonstrate assurance by examining
the LEC design (as is assumed by existing certification processes),
other approaches based on run-time monitoring and
enforcement may be effective. 