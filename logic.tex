\subsection{Decision Logic Synthesis}



We formalized this decision table and applied our synthesis tools, based on the
ACL2 theorem prover~\cite{acl2}, to synthesize high-assurance decision code
implementing the table logic.  This process used the $\verb+def-table+$
machinery described in~\cite{dasc2020}.  In particular, proofs are
automatically performed on the table to ensure completeness (some case always
applies, and an output is always produced) and unambiguity (no more than one
case applies, so a unique output is always produced).

Our software synthesis process starts with an executable specification that applies our generic table evaluation procedure \texttt{apply-table} to the specific table described above.  In this initial specification, the table is represented as a piece of data that is interpreted by \texttt{apply-table}.  Synthesis steps will specialize the table evaluation process, building in the specific table to create a cascade of conditional expressions.

We synthesize an executable Java program in two key steps:
\begin{enumerate}

\item Apply APT transformations to specialize and simplify the
  decision logic, creating an optimized, executable function: We use Kestrel's
  APT (Automated Program Transformations) \cite{apt} toolkit, built on ACL2,
  especially the \texttt{simplify} transformation, which applies simplifying
  rewrite rules from our library.
  %% Other synthesis steps
  %% include applications of our \texttt{letify}, and
  %% \texttt{lift-condition} transformations.
  %% \texttt{Letify} introduces \texttt{LET} constructs to bind variables and
  %% avoid re-computation, and \texttt{lift-condition} brings if-then-else
  %% expressions to the top level.
  Each of the APT transformation steps produces a proof showing the equivalence
  of its input and output.

\item Generate Java code using ATJ\cite{atj}: We use Kestrel's ATJ (ACL2-to-Java) Java
  code generator, built on ACL2, to generate Java code for integration with the
  Collins RTA.  Extending previous work (described in \cite{dasc2020}), ATJ now
  generates idiomatic and more performant Java code.

\end{enumerate}

For integration into the larger system, we created a hand-written wrapper for the generated Java decision code.  The wrapper receives and processes incoming plan messages from the Remain Well Clear Assessment, applies the generated decision code to the boolean plan variables, and publishes the decision to the Plan Switch.  Communication is done via ROS.

Our methodology for synthesizing the decision logic using program transformations allows for quick re-synthesis when the design changes.  One simply re-runs the synthesis tools using the new table-based specification as input.  This flexibility was demonstrated when simulation results necessitated additional conditions and constraints on plan selection.  Specifically, the table was changed to consider whether each plan's tCPA exceeds the 179 second planning horizon. This simply amounted to changing the \texttt{def-table} encoding in ACL2, with all the correctness proofs, synthesis/optimization steps, and Java code generation following immediately.

Future improvements to this methodology may include:
\begin{itemize}
\item Outfitting ATJ with proofs: As remarked in \cite{dasc2020}, this might involve either formalizing the semantics of Java or using Kestrel's Axe toolkit to lift Java bytecode into logic.
\item Code generation for other languages (C, Python, Rust): Recent work at Kestrel was able to achieve the same synthesis presented above using the ATC C code generator\cite{atc} in place of ATJ, with the additional benefit of having proofs-of-correctness for the code generation.
\item Formal synthesis of RTA-logic interface: At the time of publication, the integration of the plan selection logic with the Collins RTA requires a handwritten wrapper, which could potentially also be synthesized with proofs.
\end{itemize}
