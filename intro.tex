\section{Introduction}

Aircraft systems have requirements for airworthiness certification that present barriers to the use
of machine learning technologies such as neural networks. Showing that a component or system is
correct and does no harm through behaviors that were unintended by designers or unexpected by
operators is a critical aspect of the certification process. In a typical machine learning
application, much of the complexity and design information resides in its training data rather than
in the actual models or software produced. This means that it is generally not possible to determine
the correctness of a {\em learning enabled component (LEC)}, such as a neural network, 
by examining its implementation or tracing specific design
elements back to requirements.

Our team is developing assurance technologies that can support the use of machine learning in the
design of safety-critical aircraft systems. These capabilities have been integrated on Boeing’s
autonomy demonstrator aircraft to show that they can provide evidence of correct operation and
safety guarantees needed by real aircraft. In previous work \cite{dasc2020} we have used a run-time assurance
(RTA) architecture to ensure the safety of an autonomous neural network-based aircraft taxiing
application. In our current work we have applied RTA along with formal methods
modeling and analysis tools to an airborne collision avoidance system based on a neural network.
This system was demonstrated in-flight and shown to correctly monitor neural network operation and
intervene when needed to prevent violation of the “remain well clear” safety requirement relative to
an intruder aircraft. Thirty-six test conditions were flown with various collision course
geometries
%, with relative heading angles of 45, 90, 135, 180 (head-on), 225, 270, and 315 degrees
%leading to a pending collision, 
to test the robustness of the neural net trajectory generator and
the RTA mitigations.

The RTA architecture includes a run-time monitor that provides an independent assessment of the
avoidance flight plans generated by the neural network and a safe (but less optimal) backup planner.
The results of the assessment are evaluated by a decision logic component which selects (based on a
tabular specification of safety rules) a flight plan that will ensure safe flight. The core decision
logic code is synthesized from a formal specification, with most of the synthesis steps producing
machine-checked proofs of their correctness. The RTA architecture has been modeled in the
Architecture Analysis and Design Language (AADL) and formally analyzed to show that it maintains the
system safety requirements.

Remain-well-clear and collision avoidance capabilities are critical to safe flight of autonomous
military and commercial aircraft. These capabilities can also be valuable for
enhancing the safety of flight for piloted aircraft by providing pilot cueing. Run-time assurance
approaches, with run-time monitoring of machine learning software functions integrated with
contingency management functionality, will enable safe use of neural networks and enable new
autonomous capabilities for aircraft.

In this paper we will discuss: 
\begin{itemize} 
\item The assurance challenges associated with the use of LECs in safety-critical aircraft applications 
\item The autonomy framework and aircraft used to demonstrate the collision avoidance neural network capability 
\item The run-time assurance architecture developed to guarantee the that any potential unintended behaviors in the neural network do not impact safety
\item The formal methods assurance technologies applied within the architecture, including analysis of the architecture design and synthesis of decision logic from a formal specification
\item Flight testing, results obtained for various test scenarios, and lessons learned from the
demonstration
\end{itemize}
